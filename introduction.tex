\chapter{Introduction}


There are many Unmanned Aerial Vehicle (UAV) applications in military use such as rescue, combat, exploration, surveillance, reconnaissance, etc. In such tasks, UAVs are often required to navigate in a team (i.e., swarm). The significant advantages that multiple vehicles provide over a single one include sharing information, reducing cost, multi-functional cooperation, as well as configuring formation.

The purpose of constructing a formation in a team is to maintain the relative position of a set of vehicles so that we can prevent an inter-vehicle collision on a battlefield or maximize the group sensing area in military surveillance. In recent years, the communication network in multiple vehicle control strategies can be seen as two main categories: centralized and distributed approach. The former requires a leader or a central system to collect global information before planning a formation, while the latter allows each UAV to make its own decision based on local information. In general, the centralized approach provides a relatively more intuitive and easier way to implement in real-world scenarios, but it could become unstable when the central system of the team breaks down. For example, if a leader UAV is attacked and hurt by enemies, the communication between its followers and itself can be blocked, and the team may need to select another leader promptly to maintain its formation and mission operations.

On the other hand, despite the distributed multi-UAV approach is more stable and robust in task operations under hostile environment, a few common problems are challenging. Firstly, a consensus protocol among UAVs for internal communication is required since the path planning algorithm is conducted by each UAV dynamically and separately. Secondly, the configuration formation of a group can be more difficult to control and maintain in complex situations such as an environment with potential moving threats, a high density of unknown obstacles, or a narrow passage to reach the goal. Lastly, in the battle, it is sometimes necessary to reconfigure the formation whenever a drone strike occurs or to merge several allies into a bigger team while conducting more complex millitary operations. 

In order to minimize the computing cost and maximize the scalability for solving the above problems, we give up maintaining a specific configuration formation during the mission. Instead, we propose a more flexible configuration formation strategy (i.e., no fixed formations are negotiated beforehand) that requires only some notion of "compactness" of the configuration. In other word, our multi-UAV team aims to achieve the goals as follows:
\begin{enumerate}
\item Constructing a compact formation by consensus protocols among UAVs
\item Planning a dynamic path in a complex environment
\item Forming reconfiguration flexibly while team members change at certain situations
\end{enumerate}

Similar to a flock of birds or a school of fish heading toward the destination \citep{hubbard2004model}, each UAV flies in a direction depending on others' directions or behaviors. However, they tend to move together as a whole to protect their companions, meanwhile avoiding crashing into each other without a specific formation in order to maintain the flexibility and certain degree of autonomy. In our problem domain, we define the flexible formation as a set of random-positioned UAVs that fly compactly as a team and keep their inter-safety distance. The reason we form a compact formation is that some localization tasks require multiple sensors with closer relative positions. For example, more agents can increase the accuracy of detecting obstacles from sensor noise and widen the cone of observation \citep{anderson2008uav}. In other words, the purpose is to form more UAVs in a specific range of area.

A scoring system is used to evaluate the cost of the paths predicted by each UAV, and it then selects the minimum-cost path as the final decision. There is a tradeoff in the scoring system that is the goal cost and the formation cost. The former prompts the cost for a UAV to fly toward the destination, and the latter motivates it to stay closer to other teammates. Besides, we let the balance weight between the goal cost and the formation cost depends on the sensing radius and the safety distance of each UAV. Namely, a UAV will find the compact formation more critical when its sensing area is smaller, thus enhancing the density of the UAV around the area.

To deal with collision avoidance in different situations, we can switch the path planning algorithm among three states: flying, navigating, and waiting. Flying state is used in most scenarios, where a UAV can easily find a minimum cost among its predicted path. However, if every predicted path is blocked by obstacles, the UAV will automatically switch the algorithm to navigating state and clockwise rotate its direction in order to find another feasible path until the problem is solved. Finally, the waiting state is designed for resolving an possibility of inter-vehicle collision. While reconfiguring the formation before or after passing by obstacles, the selected path may collide with other teammates. In this case, the UAV should stop and plan its path in a busy-waiting status until the other teammates move ahead and the possibility of collision can be avoided.

The rest of the paper is organized as follows. In section 3.1, we use the dynamic windows approach \citep{580977} to dynamically plan paths for multiple UAVs under several constraints. In section 3.2, we introduce the scoring system for configuring a compact formation. In section 3.3, we summarize the distributed configuration formation control algorithm. Finally, the experimental results and future work are presented in section 4 and 5.
