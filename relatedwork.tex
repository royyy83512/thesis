\chapter{Related Work}

Preventing inter-vehicle crashing and obstacle collision is the most critical issue when forming coordination \citep{580977}. In the previous papers, the main approaches can be seen as three categories: behavioral structure, leader-follower, and virtual structure, each of which has its advantages and disadvantages.

Behavior-based structure \citep{balch1998behavior}, \citep{1261347}, which aims to design different behaviors for robots such as avoiding static obstacles, avoiding robots, moving to goal, and maintaining formation. The predetermined formations can also be switched by using graphical theory. The merit of the structure is that it requires less communication with other robots due to decentralization. However, it is unlikely to deal with problems in a more complex environment and hard to prove whether the weight of each behavior is optimal computationally. Therefore, more machine learning researches have targeted this issue \citep{bithas2019survey}.

\citet{tanner2004leader}, \citet{fredslund2002general}, \citet{luo2013uav} proposed leader-follower approaches to deal with UAVs that have limited sensing scope. Each UAV, excluding the leader, selects a neighbor to follow by a predetermined protocol. The graph of the neighbor relationship can be seen as a spanning tree, where each UAV can maintain a related position to its neighbor. Thus, it is not required to sense every other UAV's position, but the formation shape is limited due to the leader-follower structure among neighbors. 

In the virtual structure, all the UAVs have a geometric relationship based on a virtual point or virtual leader formations \citep{beard2001coordination}. Compare to the leader-follower approach, a virtual leader structure is proposed to improve its robustness since every UAV will not directly affect others' direction, but one obvious disadvantage is that reconfigure the formation is more challenging. \citet{6451251} propose a decentralized scheme based on both behavior and virtual leader structure to control the formation and avoid obstacles, respectively.

To solve the above problems of the three structures, the research about consensus theory has emerged as a challenging topic in recent years. \citet{ren2007consensus} has proved that behavior-based, leader-follower, and virtual structure can be regarded as special cases of consensus theory. Therefore, many papers combined different formation control structures by designing consensus protocols. \citet{7419717} use consensus variables calculated among a UAV and its neighbors to stabilize the formation. \citet{8295262} propose a multiple-leaders scheme, where all leaders should reach a consensus before configuring a formation, and the followers then converge to the convex hull formed by those of leaders. However, the two methods are not able to deal with problems such as formation splitting and merging. Thus, \citet{7487747}, \citet{8793765} proposed a decentralized base formation control in dynamic environment. With different predetermined formation, the team can reach a consensus when encountering obstacles or a narrow trail. A consensus protocol also decides formation splitting or merging. \citet{5069925} propose a virtual leader approach, which only requires local knowledge given by a UAV's neighbors. Each UAV only needs to maintain the relative position to keep the formation.

Although many researchers have investigated the formation control problems, few works focus on reconfiguration issue or a relatively more complex environment of high uncertainty. Based on \citep{5069925}, we propose a virtual center of gravity as a virtual leader that we can flexibly adjust the positions for every UAV. Therefore, our work allows UAV teams to avoid obstacles and inter-vehicle collision as the priority and maintain distance among teammates in all situations.
